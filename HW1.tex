% Options for packages loaded elsewhere
\PassOptionsToPackage{unicode}{hyperref}
\PassOptionsToPackage{hyphens}{url}
%
\documentclass[
]{article}
\usepackage{lmodern}
\usepackage{amssymb,amsmath}
\usepackage{ifxetex,ifluatex}
\ifnum 0\ifxetex 1\fi\ifluatex 1\fi=0 % if pdftex
  \usepackage[T1]{fontenc}
  \usepackage[utf8]{inputenc}
  \usepackage{textcomp} % provide euro and other symbols
\else % if luatex or xetex
  \usepackage{unicode-math}
  \defaultfontfeatures{Scale=MatchLowercase}
  \defaultfontfeatures[\rmfamily]{Ligatures=TeX,Scale=1}
\fi
% Use upquote if available, for straight quotes in verbatim environments
\IfFileExists{upquote.sty}{\usepackage{upquote}}{}
\IfFileExists{microtype.sty}{% use microtype if available
  \usepackage[]{microtype}
  \UseMicrotypeSet[protrusion]{basicmath} % disable protrusion for tt fonts
}{}
\makeatletter
\@ifundefined{KOMAClassName}{% if non-KOMA class
  \IfFileExists{parskip.sty}{%
    \usepackage{parskip}
  }{% else
    \setlength{\parindent}{0pt}
    \setlength{\parskip}{6pt plus 2pt minus 1pt}}
}{% if KOMA class
  \KOMAoptions{parskip=half}}
\makeatother
\usepackage{xcolor}
\IfFileExists{xurl.sty}{\usepackage{xurl}}{} % add URL line breaks if available
\IfFileExists{bookmark.sty}{\usepackage{bookmark}}{\usepackage{hyperref}}
\hypersetup{
  pdftitle={Homework Assignment 1},
  pdfauthor={Michael Kalmus},
  hidelinks,
  pdfcreator={LaTeX via pandoc}}
\urlstyle{same} % disable monospaced font for URLs
\usepackage[margin=1in]{geometry}
\usepackage{color}
\usepackage{fancyvrb}
\newcommand{\VerbBar}{|}
\newcommand{\VERB}{\Verb[commandchars=\\\{\}]}
\DefineVerbatimEnvironment{Highlighting}{Verbatim}{commandchars=\\\{\}}
% Add ',fontsize=\small' for more characters per line
\usepackage{framed}
\definecolor{shadecolor}{RGB}{248,248,248}
\newenvironment{Shaded}{\begin{snugshade}}{\end{snugshade}}
\newcommand{\AlertTok}[1]{\textcolor[rgb]{0.94,0.16,0.16}{#1}}
\newcommand{\AnnotationTok}[1]{\textcolor[rgb]{0.56,0.35,0.01}{\textbf{\textit{#1}}}}
\newcommand{\AttributeTok}[1]{\textcolor[rgb]{0.77,0.63,0.00}{#1}}
\newcommand{\BaseNTok}[1]{\textcolor[rgb]{0.00,0.00,0.81}{#1}}
\newcommand{\BuiltInTok}[1]{#1}
\newcommand{\CharTok}[1]{\textcolor[rgb]{0.31,0.60,0.02}{#1}}
\newcommand{\CommentTok}[1]{\textcolor[rgb]{0.56,0.35,0.01}{\textit{#1}}}
\newcommand{\CommentVarTok}[1]{\textcolor[rgb]{0.56,0.35,0.01}{\textbf{\textit{#1}}}}
\newcommand{\ConstantTok}[1]{\textcolor[rgb]{0.00,0.00,0.00}{#1}}
\newcommand{\ControlFlowTok}[1]{\textcolor[rgb]{0.13,0.29,0.53}{\textbf{#1}}}
\newcommand{\DataTypeTok}[1]{\textcolor[rgb]{0.13,0.29,0.53}{#1}}
\newcommand{\DecValTok}[1]{\textcolor[rgb]{0.00,0.00,0.81}{#1}}
\newcommand{\DocumentationTok}[1]{\textcolor[rgb]{0.56,0.35,0.01}{\textbf{\textit{#1}}}}
\newcommand{\ErrorTok}[1]{\textcolor[rgb]{0.64,0.00,0.00}{\textbf{#1}}}
\newcommand{\ExtensionTok}[1]{#1}
\newcommand{\FloatTok}[1]{\textcolor[rgb]{0.00,0.00,0.81}{#1}}
\newcommand{\FunctionTok}[1]{\textcolor[rgb]{0.00,0.00,0.00}{#1}}
\newcommand{\ImportTok}[1]{#1}
\newcommand{\InformationTok}[1]{\textcolor[rgb]{0.56,0.35,0.01}{\textbf{\textit{#1}}}}
\newcommand{\KeywordTok}[1]{\textcolor[rgb]{0.13,0.29,0.53}{\textbf{#1}}}
\newcommand{\NormalTok}[1]{#1}
\newcommand{\OperatorTok}[1]{\textcolor[rgb]{0.81,0.36,0.00}{\textbf{#1}}}
\newcommand{\OtherTok}[1]{\textcolor[rgb]{0.56,0.35,0.01}{#1}}
\newcommand{\PreprocessorTok}[1]{\textcolor[rgb]{0.56,0.35,0.01}{\textit{#1}}}
\newcommand{\RegionMarkerTok}[1]{#1}
\newcommand{\SpecialCharTok}[1]{\textcolor[rgb]{0.00,0.00,0.00}{#1}}
\newcommand{\SpecialStringTok}[1]{\textcolor[rgb]{0.31,0.60,0.02}{#1}}
\newcommand{\StringTok}[1]{\textcolor[rgb]{0.31,0.60,0.02}{#1}}
\newcommand{\VariableTok}[1]{\textcolor[rgb]{0.00,0.00,0.00}{#1}}
\newcommand{\VerbatimStringTok}[1]{\textcolor[rgb]{0.31,0.60,0.02}{#1}}
\newcommand{\WarningTok}[1]{\textcolor[rgb]{0.56,0.35,0.01}{\textbf{\textit{#1}}}}
\usepackage{graphicx,grffile}
\makeatletter
\def\maxwidth{\ifdim\Gin@nat@width>\linewidth\linewidth\else\Gin@nat@width\fi}
\def\maxheight{\ifdim\Gin@nat@height>\textheight\textheight\else\Gin@nat@height\fi}
\makeatother
% Scale images if necessary, so that they will not overflow the page
% margins by default, and it is still possible to overwrite the defaults
% using explicit options in \includegraphics[width, height, ...]{}
\setkeys{Gin}{width=\maxwidth,height=\maxheight,keepaspectratio}
% Set default figure placement to htbp
\makeatletter
\def\fps@figure{htbp}
\makeatother
\setlength{\emergencystretch}{3em} % prevent overfull lines
\providecommand{\tightlist}{%
  \setlength{\itemsep}{0pt}\setlength{\parskip}{0pt}}
\setcounter{secnumdepth}{-\maxdimen} % remove section numbering

\title{Homework Assignment 1}
\author{Michael Kalmus}
\date{Assigned: Oct 24, 2020, Due Sun Nov 01, 2020 11:59PM}

\begin{document}
\maketitle

\hypertarget{hw-1-instructions}{%
\subsection{HW-1 Instructions}\label{hw-1-instructions}}

\begin{quote}
Our class emphasizes clear communication of data analysis results to
non-technical audiences. I expect your HTML output documents to be
readable and well formatted. I expect you to put ample comments in your
R code to make the code understandable. Along with accuracy of results,
I am looking for quality of presentation as well. This homework is due
by \textbf{11:59PM on Nov 01st}. To complete this assignment, follow
these steps:
\end{quote}

\begin{enumerate}
\def\labelenumi{\arabic{enumi}.}
\item
  Create a new RStudio Project for this HW. Sync the project directory
  with a GitHub Repository (see instructions on Canvas on how to do
  that).
\item
  Download the \texttt{HW1.Rmd} file from Canvas. Save the file in the
  RStudio Project Directory.
\item
  Open \texttt{HW1.Rmd} in RStudio. Replace the ``Your Name Here'' text
  in the \texttt{author:} field with your name.
\item
  Supply your solutions to the homework by editing \texttt{HW1.Rmd}.
\item
  Run your code in the Console and Knit HTML frequently to check for
  errors.
\item
  You may find it easier to solve a problem by interacting only with the
  Console at first.
\item
  When you have completed the homework please check that your knits
  correctly when you click \texttt{Knit\ HTML}.
\item
  ``Push'' all your local changes to the GitHub Repo for the project.
\item
  Submit your RMD file, the HTML output and a link to your GitHub Repo
  on Canvas.
\end{enumerate}

\hypertarget{data-frame-basics}{%
\subsection{Data frame basics}\label{data-frame-basics}}

We will continue working with the nycflights dataset we looked at in
class. \textbf{Please be sure to keep the data file in the same
directory as the RMD file - your project directory.}

First - make sure that you are reading the data correctly and doing all
the data cleaning steps that we did in class. Feel free to go beyond our
class work in cleaning the data if you feel the need.

I now would like you to answer the following - all in their own separate
R code blocks.

\hypertarget{data-exploration}{%
\subsubsection{Data Exploration}\label{data-exploration}}

Let's first do some simple exploration of this data.

\begin{itemize}
\tightlist
\item
  How many airlines are there? (Hint: \texttt{levels} and
  \texttt{length} can be useful here)
\end{itemize}

\begin{Shaded}
\begin{Highlighting}[]
\KeywordTok{library}\NormalTok{(dplyr)}
\end{Highlighting}
\end{Shaded}

\begin{verbatim}
## 
## Attaching package: 'dplyr'
\end{verbatim}

\begin{verbatim}
## The following objects are masked from 'package:stats':
## 
##     filter, lag
\end{verbatim}

\begin{verbatim}
## The following objects are masked from 'package:base':
## 
##     intersect, setdiff, setequal, union
\end{verbatim}

\begin{Shaded}
\begin{Highlighting}[]
\KeywordTok{library}\NormalTok{(ggplot2)}
\end{Highlighting}
\end{Shaded}

\begin{Shaded}
\begin{Highlighting}[]
\CommentTok{# Load data}
\NormalTok{flights <-}\StringTok{ }\KeywordTok{read.csv}\NormalTok{(}\StringTok{"nycflights.csv"}\NormalTok{)}
\end{Highlighting}
\end{Shaded}

\begin{Shaded}
\begin{Highlighting}[]
\NormalTok{flights}\OperatorTok{$}\NormalTok{X <-}\StringTok{ }\OtherTok{NULL}

\CommentTok{# Factorize columns that need to be a factor}
\NormalTok{flights}\OperatorTok{$}\NormalTok{origin <-}\StringTok{ }\KeywordTok{as.factor}\NormalTok{(flights}\OperatorTok{$}\NormalTok{origin)}
\NormalTok{flights}\OperatorTok{$}\NormalTok{tailnum <-}\StringTok{ }\KeywordTok{as.factor}\NormalTok{(flights}\OperatorTok{$}\NormalTok{tailnum)}
\NormalTok{flights}\OperatorTok{$}\NormalTok{month <-}\StringTok{ }\KeywordTok{as.factor}\NormalTok{(flights}\OperatorTok{$}\NormalTok{month)}
\NormalTok{flights}\OperatorTok{$}\NormalTok{dest <-}\StringTok{ }\KeywordTok{as.factor}\NormalTok{(flights}\OperatorTok{$}\NormalTok{dest)}
\NormalTok{flights}\OperatorTok{$}\NormalTok{carrier <-}\StringTok{ }\KeywordTok{as.factor}\NormalTok{(flights}\OperatorTok{$}\NormalTok{carrier)}
\NormalTok{flights}\OperatorTok{$}\NormalTok{flight <-}\StringTok{ }\KeywordTok{as.factor}\NormalTok{(flights}\OperatorTok{$}\NormalTok{flight)}

\CommentTok{# Re-check data}
\KeywordTok{str}\NormalTok{(flights)}
\end{Highlighting}
\end{Shaded}

\begin{verbatim}
## 'data.frame':    336776 obs. of  19 variables:
##  $ year          : int  2013 2013 2013 2013 2013 2013 2013 2013 2013 2013 ...
##  $ month         : Factor w/ 12 levels "1","2","3","4",..: 1 1 1 1 1 1 1 1 1 1 ...
##  $ day           : int  1 1 1 1 1 1 1 1 1 1 ...
##  $ dep_time      : int  517 533 542 544 554 554 555 557 557 558 ...
##  $ sched_dep_time: int  515 529 540 545 600 558 600 600 600 600 ...
##  $ dep_delay     : int  2 4 2 -1 -6 -4 -5 -3 -3 -2 ...
##  $ arr_time      : int  830 850 923 1004 812 740 913 709 838 753 ...
##  $ sched_arr_time: int  819 830 850 1022 837 728 854 723 846 745 ...
##  $ arr_delay     : int  11 20 33 -18 -25 12 19 -14 -8 8 ...
##  $ carrier       : Factor w/ 16 levels "9E","AA","AS",..: 12 12 2 4 5 12 4 6 4 2 ...
##  $ flight        : Factor w/ 3844 levels "1","2","3","4",..: 1382 1545 1042 677 425 1527 469 3700 69 266 ...
##  $ tailnum       : Factor w/ 4043 levels "D942DN","N0EGMQ",..: 180 524 2401 3204 2661 1142 1829 3300 2207 1178 ...
##  $ origin        : Factor w/ 3 levels "EWR","JFK","LGA": 1 3 2 2 3 1 1 3 2 3 ...
##  $ dest          : Factor w/ 105 levels "ABQ","ACK","ALB",..: 44 44 59 13 5 70 36 43 55 70 ...
##  $ air_time      : int  227 227 160 183 116 150 158 53 140 138 ...
##  $ distance      : int  1400 1416 1089 1576 762 719 1065 229 944 733 ...
##  $ hour          : int  5 5 5 5 6 5 6 6 6 6 ...
##  $ minute        : int  15 29 40 45 0 58 0 0 0 0 ...
##  $ time_hour     : chr  "2013-01-01 05:00:00" "2013-01-01 05:00:00" "2013-01-01 05:00:00" "2013-01-01 05:00:00" ...
\end{verbatim}

\begin{Shaded}
\begin{Highlighting}[]
\KeywordTok{length}\NormalTok{(}\KeywordTok{levels}\NormalTok{(flights}\OperatorTok{$}\NormalTok{carrier))}
\end{Highlighting}
\end{Shaded}

\begin{verbatim}
## [1] 16
\end{verbatim}

There are 16 airlines

\begin{itemize}
\tightlist
\item
  How many flights there were by the airline with code \texttt{OO}?
  (Hint: \texttt{nrow} can be useful here along with logical indexing)
\end{itemize}

\begin{Shaded}
\begin{Highlighting}[]
\NormalTok{flights }\OperatorTok\StringTok{ }\KeywordTok{filter}\NormalTok{(carrier }\OperatorTok{==}\StringTok{ "OO"}\NormalTok{) }\OperatorTok\StringTok{ }\KeywordTok{nrow}\NormalTok{()}
\end{Highlighting}
\end{Shaded}

\begin{verbatim}
## [1] 32
\end{verbatim}

There are 32 flights with airline code `OO'

\begin{itemize}
\tightlist
\item
  How long is the shortest flight out of any NYC airport? (Hint:
  \texttt{min} can be useful, remember to handle \texttt{NA} values)
\end{itemize}

\begin{Shaded}
\begin{Highlighting}[]
\KeywordTok{min}\NormalTok{(flights}\OperatorTok{$}\NormalTok{airgain, }\DataTypeTok{na.rm =} \OtherTok{TRUE}\NormalTok{)}
\end{Highlighting}
\end{Shaded}

\begin{verbatim}
## Warning in min(flights$airgain, na.rm = TRUE): no non-missing arguments to min;
## returning Inf
\end{verbatim}

\begin{verbatim}
## [1] Inf
\end{verbatim}

The shortest flight out of any NYC airport is 20 minutes.

\begin{itemize}
\tightlist
\item
  How many flights where there by United Airlines (code: UA) on Jan 12th
  2013?
\end{itemize}

\begin{Shaded}
\begin{Highlighting}[]
\NormalTok{flights }\OperatorTok\StringTok{ }\KeywordTok{filter}\NormalTok{(carrier }\OperatorTok{==}\StringTok{ "UA"} \OperatorTok{&}\StringTok{ }\NormalTok{year }\OperatorTok{==}\StringTok{ "2013"} \OperatorTok{&}\StringTok{ }\NormalTok{day }\OperatorTok{==}\StringTok{ "12"} \OperatorTok{&}\StringTok{ }\NormalTok{month }\OperatorTok{==}\StringTok{ "1"}\NormalTok{) }\OperatorTok\StringTok{ }\KeywordTok{nrow}\NormalTok{()}
\end{Highlighting}
\end{Shaded}

\begin{verbatim}
## [1] 112
\end{verbatim}

There were 112 flights by United Airlines on Jan 12th, 2013.

\hypertarget{arrival-delay}{%
\subsubsection{Arrival Delay}\label{arrival-delay}}

Lets focus on Arrival Delay.

\begin{itemize}
\tightlist
\item
  What was the average arrival delay for all airports and all airlines
  combined in Jan 2013?
\end{itemize}

\begin{Shaded}
\begin{Highlighting}[]
\NormalTok{flights }\OperatorTok
\StringTok{  }\KeywordTok{filter}\NormalTok{(month }\OperatorTok{==}\StringTok{ "1"} \OperatorTok{&}\StringTok{ }\NormalTok{year }\OperatorTok{==}\StringTok{ "2013"}\NormalTok{) }\OperatorTok\StringTok{ }
\StringTok{  }\KeywordTok{select}\NormalTok{(arr_delay, dep_delay) }\OperatorTok
\StringTok{  }\KeywordTok{summarise}\NormalTok{(}
    \DataTypeTok{arr =} \KeywordTok{mean}\NormalTok{(arr_delay, }\DataTypeTok{na.rm =} \OtherTok{TRUE}\NormalTok{),}
    \DataTypeTok{dep =} \KeywordTok{mean}\NormalTok{(dep_delay, }\DataTypeTok{na.rm =} \OtherTok{TRUE}\NormalTok{)) }
\end{Highlighting}
\end{Shaded}

\begin{verbatim}
##        arr      dep
## 1 6.129972 10.03667
\end{verbatim}

The average arrival delay for all airports and all airlines combined in
January 2013 is 6.13 minutes.

\begin{itemize}
\tightlist
\item
  Whats was the median arrival delay for all airports and all airlines
  combined in Jan 2013?
\end{itemize}

\begin{Shaded}
\begin{Highlighting}[]
\NormalTok{flights }\OperatorTok
\StringTok{  }\KeywordTok{filter}\NormalTok{(month }\OperatorTok{==}\StringTok{ "1"} \OperatorTok{&}\StringTok{ }\NormalTok{year }\OperatorTok{==}\StringTok{ "2013"}\NormalTok{) }\OperatorTok\StringTok{ }
\StringTok{  }\KeywordTok{select}\NormalTok{(arr_delay, dep_delay) }\OperatorTok
\StringTok{  }\KeywordTok{summarise}\NormalTok{(}
    \DataTypeTok{arr =} \KeywordTok{median}\NormalTok{(arr_delay, }\DataTypeTok{na.rm =} \OtherTok{TRUE}\NormalTok{))}
\end{Highlighting}
\end{Shaded}

\begin{verbatim}
##   arr
## 1  -3
\end{verbatim}

The median arrival delay for all airports and all airlines combined in
Jan 2013 was -3.

Based on your answers to the two questions above, what can you say about
the distribution of arrival delays? Provide your answer in a text
paragraph form.

\begin{Shaded}
\begin{Highlighting}[]
\NormalTok{flights }\OperatorTok
\StringTok{  }\KeywordTok{filter}\NormalTok{(month }\OperatorTok{==}\StringTok{ "1"} \OperatorTok{&}\StringTok{ }\NormalTok{year }\OperatorTok{==}\StringTok{ "2013"}\NormalTok{) }\OperatorTok\StringTok{ }
\StringTok{  }\KeywordTok{select}\NormalTok{(arr_delay, dep_delay) }\OperatorTok
\StringTok{  }\KeywordTok{ggplot}\NormalTok{() }\OperatorTok{+}\StringTok{ }\KeywordTok{geom_histogram}\NormalTok{(}\KeywordTok{aes}\NormalTok{(}\DataTypeTok{x =}\NormalTok{ arr_delay), }\DataTypeTok{color =} \StringTok{"red"}\NormalTok{) }\OperatorTok{+}\StringTok{ }\KeywordTok{xlim}\NormalTok{(}\OperatorTok{-}\DecValTok{50}\NormalTok{, }\DecValTok{250}\NormalTok{)}
\end{Highlighting}
\end{Shaded}

\begin{verbatim}
## `stat_bin()` using `bins = 30`. Pick better value with `binwidth`.
\end{verbatim}

\begin{verbatim}
## Warning: Removed 734 rows containing non-finite values (stat_bin).
\end{verbatim}

\begin{verbatim}
## Warning: Removed 2 rows containing missing values (geom_bar).
\end{verbatim}

\includegraphics{HW1_files/figure-latex/unnamed-chunk-10-1.pdf}

Since the median is below the mean, the data for arrival delays is
skewed to the left.

\hypertarget{airline-performance}{%
\subsubsection{Airline Performance}\label{airline-performance}}

Lets see if all airlines are equally terrible as far as flight arrival
delays are concerned. For this question you will have to make sure that
airline column is coded as a factor.

\begin{itemize}
\tightlist
\item
  Calculate average arrival delays by airline (Hint: look up the command
  \texttt{tapply})
\end{itemize}

\begin{Shaded}
\begin{Highlighting}[]
\NormalTok{df2 <-}\StringTok{ }\NormalTok{flights }\OperatorTok\StringTok{ }\KeywordTok{group_by}\NormalTok{(carrier) }\OperatorTok\StringTok{ }\KeywordTok{summarize}\NormalTok{(}\DataTypeTok{mean_delay =} \KeywordTok{mean}\NormalTok{(arr_delay, }\DataTypeTok{na.rm =} \OtherTok{TRUE}\NormalTok{)) }\OperatorTok\StringTok{ }\KeywordTok{arrange}\NormalTok{(}\KeywordTok{desc}\NormalTok{(mean_delay))}
\end{Highlighting}
\end{Shaded}

\begin{verbatim}
## `summarise()` ungrouping output (override with `.groups` argument)
\end{verbatim}

\begin{Shaded}
\begin{Highlighting}[]
\NormalTok{df2}
\end{Highlighting}
\end{Shaded}

\begin{verbatim}
## # A tibble: 16 x 2
##    carrier mean_delay
##    <fct>        <dbl>
##  1 F9          21.9  
##  2 FL          20.1  
##  3 EV          15.8  
##  4 YV          15.6  
##  5 OO          11.9  
##  6 MQ          10.8  
##  7 WN           9.65 
##  8 B6           9.46 
##  9 9E           7.38 
## 10 UA           3.56 
## 11 US           2.13 
## 12 VX           1.76 
## 13 DL           1.64 
## 14 AA           0.364
## 15 HA          -6.92 
## 16 AS          -9.93
\end{verbatim}

\begin{itemize}
\tightlist
\item
  Draw a Bar Plot of Average Arrival Delays for all the Airlines (Hint:
  command for making a Bar Plot is simply \texttt{barplot})
\end{itemize}

\begin{Shaded}
\begin{Highlighting}[]
\KeywordTok{barplot}\NormalTok{(df2}\OperatorTok{$}\NormalTok{mean_delay)}
\end{Highlighting}
\end{Shaded}

\includegraphics{HW1_files/figure-latex/unnamed-chunk-12-1.pdf}

\begin{Shaded}
\begin{Highlighting}[]
\NormalTok{df2 }\OperatorTok\StringTok{ }\KeywordTok{ggplot}\NormalTok{() }\OperatorTok{+}\StringTok{ }\KeywordTok{geom_col}\NormalTok{(}\KeywordTok{aes}\NormalTok{(}\DataTypeTok{x =}\NormalTok{ carrier, }\DataTypeTok{y =}\NormalTok{ mean_delay))}
\end{Highlighting}
\end{Shaded}

\includegraphics{HW1_files/figure-latex/unnamed-chunk-12-2.pdf}

\begin{itemize}
\tightlist
\item
  Which airline has the highest average arrival delay? Which airline has
  the smallest average arrival delay? Are there airlines that actually
  have negative average delay? Provide answer to this question in a text
  paragraph form using \textbf{inline R code}.
\end{itemize}

\begin{Shaded}
\begin{Highlighting}[]
\NormalTok{df2 }\OperatorTok\StringTok{ }\KeywordTok{filter}\NormalTok{(mean_delay }\OperatorTok{==}\StringTok{ }\KeywordTok{max}\NormalTok{(mean_delay))}
\end{Highlighting}
\end{Shaded}

\begin{verbatim}
## # A tibble: 1 x 2
##   carrier mean_delay
##   <fct>        <dbl>
## 1 F9            21.9
\end{verbatim}

\begin{Shaded}
\begin{Highlighting}[]
\NormalTok{df2 }\OperatorTok\StringTok{ }\KeywordTok{filter}\NormalTok{(mean_delay }\OperatorTok{==}\StringTok{ }\KeywordTok{min}\NormalTok{(mean_delay))}
\end{Highlighting}
\end{Shaded}

\begin{verbatim}
## # A tibble: 1 x 2
##   carrier mean_delay
##   <fct>        <dbl>
## 1 AS           -9.93
\end{verbatim}

\hypertarget{air-gain}{%
\subsubsection{Air Gain}\label{air-gain}}

Create a new column named airgain such that airgain = (departure delay -
arrival delay) : this is the amount of delay a flight made up while in
air.

\begin{Shaded}
\begin{Highlighting}[]
\NormalTok{flights <-}\StringTok{ }\NormalTok{flights }\OperatorTok\StringTok{ }\KeywordTok{mutate}\NormalTok{(}\DataTypeTok{airgain =}\NormalTok{ dep_delay }\OperatorTok{-}\StringTok{ }\NormalTok{arr_delay)}
\end{Highlighting}
\end{Shaded}

\begin{enumerate}
\def\labelenumi{\alph{enumi})}
\tightlist
\item
  Explore airgain data - calculate suitable descriptive statistics and
  appropriate graphics to better understand this data. This part is open
  ended - you do what you feel works best for you.
\end{enumerate}

\begin{Shaded}
\begin{Highlighting}[]
\KeywordTok{min}\NormalTok{(flights}\OperatorTok{$}\NormalTok{airgain, }\DataTypeTok{na.rm =} \OtherTok{TRUE}\NormalTok{)}
\end{Highlighting}
\end{Shaded}

\begin{verbatim}
## [1] -196
\end{verbatim}

\begin{Shaded}
\begin{Highlighting}[]
\KeywordTok{max}\NormalTok{(flights}\OperatorTok{$}\NormalTok{airgain, }\DataTypeTok{na.rm =} \OtherTok{TRUE}\NormalTok{)}
\end{Highlighting}
\end{Shaded}

\begin{verbatim}
## [1] 109
\end{verbatim}

\begin{Shaded}
\begin{Highlighting}[]
\KeywordTok{mean}\NormalTok{(flights}\OperatorTok{$}\NormalTok{airgain, }\DataTypeTok{na.rm =} \OtherTok{TRUE}\NormalTok{)}
\end{Highlighting}
\end{Shaded}

\begin{verbatim}
## [1] 5.659779
\end{verbatim}

\begin{Shaded}
\begin{Highlighting}[]
\KeywordTok{median}\NormalTok{(flights}\OperatorTok{$}\NormalTok{airgain, }\DataTypeTok{na.rm =} \OtherTok{TRUE}\NormalTok{)}
\end{Highlighting}
\end{Shaded}

\begin{verbatim}
## [1] 7
\end{verbatim}

\begin{Shaded}
\begin{Highlighting}[]
\KeywordTok{summary}\NormalTok{(flights}\OperatorTok{$}\NormalTok{airgain)}
\end{Highlighting}
\end{Shaded}

\begin{verbatim}
##    Min. 1st Qu.  Median    Mean 3rd Qu.    Max.    NA's 
## -196.00   -3.00    7.00    5.66   17.00  109.00    9430
\end{verbatim}

\begin{Shaded}
\begin{Highlighting}[]
\NormalTok{flights }\OperatorTok\StringTok{ }\KeywordTok{ggplot}\NormalTok{() }\OperatorTok{+}\StringTok{ }\KeywordTok{geom_boxplot}\NormalTok{(}\KeywordTok{aes}\NormalTok{(}\DataTypeTok{x =}\NormalTok{ carrier, }\DataTypeTok{y =}\NormalTok{ airgain))}
\end{Highlighting}
\end{Shaded}

\begin{verbatim}
## Warning: Removed 9430 rows containing non-finite values (stat_boxplot).
\end{verbatim}

\includegraphics{HW1_files/figure-latex/unnamed-chunk-15-1.pdf}

\begin{enumerate}
\def\labelenumi{\alph{enumi})}
\setcounter{enumi}{1}
\tightlist
\item
  Answer the questions:
\end{enumerate}

\begin{itemize}
\tightlist
\item
  do airlines actually gain any time when in air on average?
\end{itemize}

\begin{Shaded}
\begin{Highlighting}[]
\KeywordTok{mean}\NormalTok{(flights}\OperatorTok{$}\NormalTok{airgain, }\DataTypeTok{na.rm =} \OtherTok{TRUE}\NormalTok{)}
\end{Highlighting}
\end{Shaded}

\begin{verbatim}
## [1] 5.659779
\end{verbatim}

\begin{itemize}
\tightlist
\item
  Calculate average airgain for different airlines - which airlines do a
  better job, which do a worse job?
\end{itemize}

\begin{Shaded}
\begin{Highlighting}[]
\NormalTok{flights }\OperatorTok\StringTok{ }
\StringTok{  }\KeywordTok{group_by}\NormalTok{(carrier) }\OperatorTok\StringTok{ }
\StringTok{  }\KeywordTok{summarize}\NormalTok{(}\DataTypeTok{mean_gain =} \KeywordTok{mean}\NormalTok{(airgain, }\DataTypeTok{na.rm =} \OtherTok{TRUE}\NormalTok{)) }\OperatorTok\StringTok{ }
\StringTok{  }\KeywordTok{arrange}\NormalTok{(}\KeywordTok{desc}\NormalTok{(mean_gain)) }\OperatorTok\StringTok{ }
\StringTok{  }\KeywordTok{filter}\NormalTok{(mean_gain }\OperatorTok{==}\StringTok{ }\KeywordTok{max}\NormalTok{(mean_gain) }\OperatorTok{|}\StringTok{ }\NormalTok{mean_gain }\OperatorTok{==}\StringTok{ }\KeywordTok{min}\NormalTok{(mean_gain))}
\end{Highlighting}
\end{Shaded}

\begin{verbatim}
## `summarise()` ungrouping output (override with `.groups` argument)
\end{verbatim}

\begin{verbatim}
## # A tibble: 2 x 2
##   carrier mean_gain
##   <fct>       <dbl>
## 1 AS          15.8 
## 2 F9          -1.72
\end{verbatim}

From the dataframe, we can see that carrier AS does the best job,
gaining 15.76 mins of airtime on average, while F9 does the worst,
losing 1.72 mins of airtime on average.

\hypertarget{merging-data-frames}{%
\subsubsection{Merging Data Frames}\label{merging-data-frames}}

\begin{quote}
This section and the next is new compared to the class exercise. As you
had an opportunity to work together in your breakout rooms for previous
questions, this and the next section will carry a higher weight in
grading for this HW.
\end{quote}

You can get detailed information about the physical planes in our
dataset in this file: \texttt{planes.csv}. Download and save this file
in your project directory.

\begin{enumerate}
\def\labelenumi{\alph{enumi})}
\tightlist
\item
  Read the \texttt{planes.csv} file using \texttt{read.csv} command. Do
  any data cleaning necessary.
\end{enumerate}

\begin{Shaded}
\begin{Highlighting}[]
\NormalTok{planes <-}\StringTok{ }\KeywordTok{read.csv}\NormalTok{(}\StringTok{"planes.csv"}\NormalTok{)}
\end{Highlighting}
\end{Shaded}

\begin{Shaded}
\begin{Highlighting}[]
\NormalTok{planes}\OperatorTok{$}\NormalTok{X <-}\StringTok{ }\OtherTok{NULL}
\NormalTok{planes}\OperatorTok{$}\NormalTok{tailnum <-}\StringTok{ }\KeywordTok{factor}\NormalTok{(planes}\OperatorTok{$}\NormalTok{tailnum)}
\NormalTok{planes}\OperatorTok{$}\NormalTok{type <-}\StringTok{ }\KeywordTok{factor}\NormalTok{(planes}\OperatorTok{$}\NormalTok{type)}
\NormalTok{planes}\OperatorTok{$}\NormalTok{manufacturer <-}\StringTok{ }\KeywordTok{factor}\NormalTok{(planes}\OperatorTok{$}\NormalTok{manufacturer)}
\NormalTok{planes}\OperatorTok{$}\NormalTok{model <-}\StringTok{ }\KeywordTok{factor}\NormalTok{(planes}\OperatorTok{$}\NormalTok{model)}
\NormalTok{planes}\OperatorTok{$}\NormalTok{engine <-}\StringTok{ }\KeywordTok{factor}\NormalTok{(planes}\OperatorTok{$}\NormalTok{engine)}
\NormalTok{planes}\OperatorTok{$}\NormalTok{year <-}\StringTok{ }\KeywordTok{factor}\NormalTok{(planes}\OperatorTok{$}\NormalTok{year)}
\end{Highlighting}
\end{Shaded}

\begin{enumerate}
\def\labelenumi{\alph{enumi})}
\setcounter{enumi}{1}
\tightlist
\item
  Merge the flights data and the planes data using the \texttt{merge}
  command. You should do the merge on the common column named
  \texttt{tailnum}. \emph{getting this right may need some trial and
  error and getting some help}.
\end{enumerate}

\begin{Shaded}
\begin{Highlighting}[]
\NormalTok{merge_df <-}\StringTok{ }\KeywordTok{merge}\NormalTok{(flights, planes, }\DataTypeTok{sort =} \OtherTok{FALSE}\NormalTok{, }\DataTypeTok{by =} \StringTok{"tailnum"}\NormalTok{)}
\end{Highlighting}
\end{Shaded}

\begin{enumerate}
\def\labelenumi{\alph{enumi})}
\setcounter{enumi}{2}
\tightlist
\item
  Now that you have a merged dataset, think of what interesting
  questions that you can ask that can be answered using the merged
  dataset. You are asked to pose five interesting questions and answer
  them. (For example: who are the top 10 manufacturers of planes that
  fly out of NYC airports?) \textbf{Be creative. Be bold. Ask questions
  that you would want to know answers to even if you were not doing this
  for a HW. }
\end{enumerate}

Q1: What engine types

\hypertarget{making-your-html-look-nice}{%
\subsubsection{Making Your HTML Look
Nice}\label{making-your-html-look-nice}}

We want our report to be good looking, professional documents. To that
end, I am asking you to do the following:

\begin{itemize}
\item
  Have a floating table of contents
\item
  Include code folding in your output. You can find more about code
  folding here:
  \url{https://bookdown.org/yihui/rmarkdown/html-document.html\#code-folding}
\end{itemize}

That's it. Once you are done, make sure everything works and knits well
and then you can push your changes to the GitHub repo and uplaod the RMD
flile and the html output to Canvas.

\textbf{Have Fun!}

Sanjeev

\end{document}
